\PassOptionsToPackage{unicode}{hyperref}
\documentclass[
  ukrainian,
  14pt
]{extreport}
\usepackage{lmodern}
\usepackage{hyperref}
\makeatletter
\hypersetup{
    colorlinks=true,
    linkcolor=blue,
    filecolor=magenta,
    urlcolor=cyan,
}
\makeatother
\usepackage{amssymb,amsmath,amsthm,url}
\usepackage[margin=2cm]{geometry}
\usepackage{longtable,booktabs}
\usepackage{etoolbox}
\usepackage{titling}
\usepackage{graphicx}
\usepackage{float}
\usepackage[dvipsnames]{xcolor}
\usepackage[ukrainian]{babel}
\usepackage{setspace}
\usepackage{xcolor}
\usepackage{multirow}
\usepackage{comment}
\usepackage{booktabs}
\usepackage{tikz}
\setcounter{secnumdepth}{-1} 
\usepackage{unicode-math}
  \defaultfontfeatures{Scale=MatchLowercase}
  \defaultfontfeatures[\rmfamily]{Ligatures=TeX,Scale=1}
  \setmainfont[]{Times New Roman}
  \setsansfont[]{Arial}
  \setmonofont[]{Consolas}
  \makeatother
\usepackage[labelsep=period]{caption}
\usepackage{subcaption}

\author{}
\title{\Huge Лабораторна робота №2 \\\Large Дослідження ВАХ діодів}
\date{}
             
\begin{document}
\begin{titlepage} 
	\newcommand{\HRule}{\rule{\linewidth}{0.5mm}} 
	
	\center 
	
	\textsc{\Large МІНІСТЕРСТВО ОСВІТИ І НАУКИ УКРАЇНИ\\ \Large КИЇВСЬКИЙ НАЦІОНАЛЬНИЙ УНІВЕРСИТЕТ ІМЕНІ ТАРАСА ШЕВЧЕНКА}\\[1.5cm] 

	
	\HRule\\[0.4cm]
	
	{\huge \bfseries  Лабораторна робота №2 \\\Large \bfseries Моделювання пасивних RC фільтрів
    }\\[0.4cm]
	
	\HRule\\[1.5cm]

	
	

	{\large\textit{Автор}}\\
	\large Столяров Андрій Дмитрович, \\\large група 5-А, Фізичний Факультет 
	
	
	\vfill\vfill\vfill 
	\vfill
	{\normalsize Київ, \today} 
\end{titlepage}
\tableofcontents
\chapter{Вступ}
В даній роботі моделюють такі пасивні RC
фільтри: ФНЧ, ФВЧ, смуговий та
загороджувальний. Робота виконувалась у програмі \textbf{Multisim14}.
\end{document}